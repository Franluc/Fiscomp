% ****** Start of file apssamp.tex ******
%
%   This file is part of the APS files in the REVTeX 4.1 distribution.
%   Version 4.1r of REVTeX, August 2010
%
%   Copyright (c) 2009, 2010 The American Physical Society.
%
%   See the REVTeX 4 README file for restrictions and more information.
%
% TeX'ing this file requires that you have AMS-LaTeX 2.0 installed
% as well as the rest of the prerequisites for REVTeX 4.1
%
% See the REVTeX 4 README file
% It also requires running BibTeX. The commands are as follows:
%
%  1)  latex apssamp.tex
%  2)  bibtex apssamp
%  3)  latex apssamp.tex
%  4)  latex apssamp.tex
%
\documentclass[
 reprint,
 amsmath,amssymb,
 aps,
 a4paper
]{revtex4-1}

\usepackage[utf8]{inputenc}
\usepackage[spanish]{babel}
\usepackage{hyperref}
\usepackage{datetime}
\usepackage{graphicx}% Include figure files
\usepackage{dcolumn}% Align table columns on decimal point
\usepackage{bm}% bold math
\usepackage{float} % PARA USAR EL [H]
\usepackage{subfig} % PARA PONER SUBFIGURAS
\usepackage{listings} % PARA CODIGO
\usepackage{framed} % PARA CAJITAS
\usepackage{verbatim} % PARA COMMENTS
\lstset{ %
  basicstyle=\footnotesize,        % the size of the fonts that are used for the code
  breakatwhitespace=false,         % sets if automatic breaks should only happen at whitespace
  breaklines=true,                 % sets automatic line breaking
  }

\decimalpoint


\begin{document}

\title{Difusión en materia nuclear - Nuclear matter difussion}

\author{Franco Tavella}%
 \email{tavellafran@gmail.com}
\affiliation{%
 Departamento de F\'\i sica, Facultad de Ciencias Exactas y Naturales, Universidad de Buenos Aires,\\
 Pabell\'on I, Ciudad Universitaria, 1428 Buenos Aires, Argentina. 
}%

\author{Lucas Longo}%
 \email{lucaslongo52@gmail.com}
\affiliation{%
 Departamento de F\'\i sica, Facultad de Ciencias Exactas y Naturales, Universidad de Buenos Aires,\\
 Pabell\'on I, Ciudad Universitaria, 1428 Buenos Aires, Argentina. 
}%

\date{\today}

\begin{abstract}
ABSTRACT

\end{abstract}

\maketitle


\section{\label{seq:intro}Introducción}

Potenciales y fuerzas discretizadas que se usan en la simulación, figuras \ref{fig:pots} y \ref{fig:forcs}. Podemos observar comportamientos repulsivos para todos los casos.
\begin{figure}[H]
\centerline{
  \includegraphics[width=1.0\linewidth]{potentials.png}}
  \caption{\small Potenciales discretizados. Tenemos dos tipos, uno para particulas iguales y otro para particulas diferentes.}
  \label{fig:pots}
\end{figure}

\begin{figure}[H]
\centerline{
  \includegraphics[width=1.0\linewidth]{forces.png}}
  \caption{\small Fuerzas discretizadas. Tenemos dos tipos, uno para particulas iguales y otro para particulas diferentes.}
  \label{fig:forcs}
\end{figure}

\section{\label{seq:difu}Difusión}
La ecuación de difusión describe el comportamiento colectivo de partículas microscópicas que resulta de su movimiento aleatorio. Se puede arribar a ella a partir de la ecuación de continuidad junto con la primer ley fenomenológica de Fick.
\begin{equation}\label{eq:cont}
\frac{\partial \phi}{\partial t} + \nabla \cdot \vec{j}  = 0
\end{equation}
\begin{equation}
\vec{j} = - D(\phi,r)\nabla \phi(r,t)
\end{equation}
en donde $\phi(r,t)$ es la densidad de la sustancia que difunde y $D(\phi,r)$ el coeficiente de difusión colectivo. Para un caso isótropo y homogéneo pensamos a $D(\phi,r) = D$ como una constante y obtenemos una ecuación análoga a la del calor
\begin{equation}\label{eq:difu}
\frac{\partial \phi(r,t)}{\partial t} = D \nabla^{2}\phi(r,t)
\end{equation}

\section{\label{seq:discre}Discretización}
Para la obtención del coeficiente de difusión $D$ debemos resolver la ecuación \ref{eq:difu}. Para ello debemos medir de las simulaciones la cantidad $\phi(r,t)$. Realizaremos una discretización espacial y otra temporal. Utilizaremos los subíndices $i$, $j$ y $k$ para indicar las subceldas espaciales correspondientes a las direcciones $x$, $y$ y $z$ respectivamente. El grillado será isótropo y homogéneo. Temporalmente utilizaremos el subíndice $n$. Entonces tendremos las siguientes cantidades si dividimos cada dirección en $m$ espacios y la caja de simulación posee longitud $L$
\begin{equation}
\phi(r_{i,j,k},t_{n}) = \frac{m^{3}}{L^{3}}N_{i,j,k} 
\end{equation}
en donde $r_{i,j,k}$ representa la posición de la celda i-j-k-ésima y $N_{i,j,k}$ el número de partículas en ésta.
\section{\label{seq:msd}Mean Square Displacement}
En las figuras \ref{fig:msdneut} y \ref{fig:msdprot} vemos los desplazamientos cuadráticos medios para una simulación con protones y neutrones juntos.
\begin{figure}[H]
\centerline{
  \includegraphics[width=1.0\linewidth]{msdvst_neut.png}}
  \caption{\small MSD de Neutrones para una simulación con posiciones corregidas para eliminar las condiciones periodicas de contorno en función del tiempo. El valor de D obtenido es de $0.087 \pm 0.001$ en unidades de lj.}
  \label{fig:msdneut}
\end{figure}

La relación de Einstein para difusión establece una relación entre las posiciones de las moléculas y el coeficiente de difusión D.

\begin{equation}\label{eq:rel_einst}
2tD=\frac{1}{3}<[\vec{r}(t)-\vec{r}(t_0)]^2>
\end{equation}

Este resultado es aplicable cuando el tiempo $t$ es largo en comparación al tiempo promedio entre colisiones. Es posible entonces obtener un valor para D:

\begin{equation}\label{eq:D_einst}
D=\lim_{t \to \infty}\frac{<[\vec{r}(t)-\vec{r}(t_0)]^2>}{6t}
\end{equation} 

En las ecuaciones \ref{eq:rel_einst} y \ref{eq:D_einst} el valor $<[\vec{r}(t)-\vec{r}(t_0)]^2>$ es el desplazamiento cuadrático medio $MSD(t)$ del sistema  a tiempo $t$. El símbolo $<>$ debe entenderse como un promedio sobre el número de partículas y sobre orígenes de tiempo $t_0$.

EL coeficiente de difusión $D$ puede también ser evaluado de otras maneras. La relación de Einstein puede reescribirse como:(Ver Haile pág 302)

\begin{equation}\label{eq:D_einst_2}
D=\frac{1}{6}\lim_{t \to \infty}\frac{d}{dt}<[\vec{r}(t)-\vec{r}(t_0)]^2>
\end{equation}

La ecuación \ref{eq:D_einst_2} muestra que $D$ es proporcional a la pendiente del desplazamiento cuadrático medio a tiempos largos. La forma \ref{eq:D_einst_2} se prefiere a la forma \ref{eq:D_einst} en paticular para bajas densidades.\\

El desplazamiento cuadrático medio del sistema de $N$ partículas puede aproximarse como:

\begin{equation}\label{eq:MSD}
MSD(t)=\frac{1}{M.N}\sum_{t_0}^{M}\sum_{i}^{N}[\vec{r}_i(t_0+t)-\vec{r}_i(t_0)]^2
\end{equation}

Donde $N$ es el número de partículas y $M$ es el número de orígenes de tiempo disponibles. Para un total de $L$ pasos temporales separados por $\Delta t$, M depende del tiempo $t$ de la medición mediante: $M=L-\frac{t}{\Delta t}$. 

El algoritmo que hemos utilizado para obtener $MSD(t)$ es el siguiente (ver Haile pág 284):

1) Elegir un valor para el tiempo t

2) Loop sobre orígenes de tiempos $t_j$, $\{j=1,...,M(t)\}$

3) Para cada origen de tiempo $t_o$, loop sobre $i$, $\{i=1,...,N\}$ las $N$ partículas, para leer $\vec{r}_i(t_j+t)$ y $\vec{r}_i(t_j)$

4) Acumular $[\vec{r}_i(t_{j}+t)-\vec{r}_i(t_{j})]^2$

5) Volver al paso (1) hasta barrer todos los $t$

 nos permite hallar el coeficiente de difusión dado que:

 

\begin{figure}[H]
\centerline{
  \includegraphics[width=1.0\linewidth]{msdvst_prot.png}}
  \caption{\small MSD de Protones para una simulación con posiciones corregidas para eliminar las condiciones periodicas de contorno en función del tiempo. El valor de D obtenido es de $0.038 \pm 0.001$ en unidades de lj.}
  \label{fig:msdprot}
\end{figure}

\begin{figure}[H]
\centerline{
  \includegraphics[width=1.0\linewidth]{msd.png}}
  \caption{\small MSD para una simulación con posiciones corregidas para eliminar las condiciones periodicas de contorno en función del tiempo. timestep 0.1 dump 10 run 10000 temp 4.0}
  \label{fig:msd_t400} 
\end{figure}

A partir de la ecuación \ref{eq:D_einst} se calculo $D$ para cada tiempo $t$.

\begin{figure}[H]
\centerline{
  \includegraphics[width=1.0\linewidth]{dif.png}}
  \caption{\small Coeficiente de Difusion $D$ para protones($Z$) y neutrones ($N$)}
  \label{fig:D_t400}
\end{figure}

\section{\label{seq:vaf}Velocity Autocorrelation Function}

Las relación de Green-Kubo permite relacionar el desplazamiento cuadrático medio de una cantidad dinámica $A$ con la integral de una funcion de autocorrelación:

 \begin{equation}\label{eq:GK}
 \lim_{t \to \infty} \frac{<[A(t)-A(t_0)]^2>}{2t}=\int_{0}^{\infty}d\tau<\dot{A}(\tau)\dot{A}(\tau_0)>
\end{equation}

En particular para el coeficiente de difusión, la relación de Green-Kubo es:

 \begin{equation}\label{eq:GK_D}
 D=\frac{1}{3}\int_{0}^{\infty}dt<\vec{v}(t)\vec{v}(t_0)>
\end{equation}

Aquí $<\vec{v}(t)\vec{v}(t_0)>$ es la función de autocorrelación de velocidades $VACF(t)$. Nuevamente el símbolo $<>$ indica un promedio sobre el número de partículas y sobre orígenes de tiempos $t_0$.

Para calcular $VACF(t)$ hemos utilizado nuevamente el algoritmo ya usado para el $MSD(t)$:

\begin{equation}\label{eq:MSD}
VACF(t)=\frac{1}{M.N}\sum_{t_0}^{M}\sum_{i}^{N}\vec{v}_i(t_0).\vec{v}_i(t_0+t)
\end{equation}

\begin{figure}[H]
\centerline{
  \includegraphics[width=1.0\linewidth]{vacf_N_200.png}}
  \caption{\small Función de autocorrelación de velocidades para neutrones. timestep 0.1 dump 10 run 10000 temp 4.0}
  \label{fig:vacf_N_200}
\end{figure}

\begin{figure}[H]
\centerline{
  \includegraphics[width=1.0\linewidth]{vacf_Z_200.png}}
  \caption{\small Función de autocorrelación de velocidades para protones. timestep 0.1 dump 10 run 10000 temp 4.0}
  \label{fig:vacf_Z_200}
\end{figure}

Integracion de la funcion VACF mediante la regla de Simpson.\\
Numero de puntos: 401\\
Numero de areas de Simpson: 200\\
Integral de VACF protones=0.0435208564516\\
Integral de VACF neutrones=0.0982231090351\\



\section{\label{seq:details}Detalles técnicos}
Las posiciones de las partículas son obtenidas a través del comando $dump$ en LAMMPS. En la documentación\cite{dumplammps} sobre este comando se hace una aclaración sobre las condiciones periódicas de contorno
\begin{framed}
\textit{Because periodic boundary conditions are enforced only on timesteps when neighbor lists are rebuilt, the coordinates of an atom written to a dump file may be slightly outside the simulation box. Re-neighbor timesteps will not typically coincide with the timesteps dump snapshots are written. See the dump\_ modify pbc command if you with to force coordinates to be strictly inside the simulation box.}
\end{framed}
Unidades lj \cite{unitslammps}
\begin{framed}
\textit{For style lj, all quantities are unitless. Without loss of generality, LAMMPS sets the fundamental quantities mass, sigma, epsilon, and the Boltzmann constant = 1. The masses, distances, energies you specify are multiples of these fundamental values. The formulas relating the reduced or unitless quantity (with an asterisk) to the same quantity with units is also given. Thus you can use the mass \& sigma \& epsilon values for a specific material and convert the results from a unitless LJ simulation into physical quantities.}
\begin{comment}
mass = mass or m
distance = sigma, where x* = x / sigma
time = tau, where t* = t (epsilon / m / sigma^2)^1/2
energy = epsilon, where E* = E / epsilon
velocity = sigma/tau, where v* = v tau / sigma
force = epsilon/sigma, where f* = f sigma / epsilon
torque = epsilon, where t* = t / epsilon
temperature = reduced LJ temperature, where T* = T Kb / epsilon
pressure = reduced LJ pressure, where P* = P sigma^3 / epsilon
dynamic viscosity = reduced LJ viscosity, where eta* = eta sigma^3 / epsilon / tau
charge = reduced LJ charge, where q* = q / (4 pi perm0 sigma epsilon)^1/2
dipole = reduced LJ dipole, moment where *mu = mu / (4 pi perm0 sigma^3 epsilon)^1/2
electric field = force/charge, where E* = E (4 pi perm0 sigma epsilon)^1/2 sigma / epsilon
density = mass/volume, where rho* = rho sigma^dim
\end{comment}
\end{framed}

\begin{thebibliography}{9}

\bibitem{dumplammps}
  Sandia National Laboratories,
  \url{http://lammps.sandia.gov/doc/dump.html},
  Consultado 02/09/17
\bibitem{unitslammps}
  Sandia National Laboratories,
  \url{http://lammps.sandia.gov/doc/units.html},
  Consultado 21/09/17

\end{thebibliography}

\end{document}

